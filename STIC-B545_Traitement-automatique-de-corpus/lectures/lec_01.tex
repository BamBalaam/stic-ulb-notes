\lecture{01}{2020-10-06}
\vspace{-1.2cm}

\section{Introduction}

Suite logique du cours "STIC-B425 - Ingénierie linguistique", il s'agit d'un tout nouveau cours
de la filière (c'est la deuxième fois qu'il est donné).\\

L'accent est mis sur des grands volumes de texte et le cours va s'orienter autour d'une étude de cas précis.
On va analyser un corpus des Archives de la Ville de Bruxelles: des bulletins communaux de 1847 à 1978.
Il s'agit donc d'un cours de NLP appliqué à des corpus de textes non structurés.
Thomas Schlesser, historien et aspirant FNRS, nous présente ce corpus (voir présentation sur l'Université Virtuelle).\\

\subsection{Plan du cours}

\noindent\textbf{Collecte de données}
\begin{itemize}
\item Constitution de corpus textuels
\item Numérisation et OCR
\item Requêtes SQL et SPARQL
\item Utilisation d'APIs
\item Web scraping\\
\end{itemize}

\noindent\textbf{Nettoyage et transformation}
\begin{itemize}
\item Extract, transform, load (ETL)
\item Détection d’anomalies
\item Catégorisation de l’information
\item Langages contrôlés
\item Standards\\
\end{itemize}

\noindent\textbf{Enrichissement et TAL}
\begin{itemize}
\item Extraction de mots-clés
\item Reconnaissance d'entités nommées
\item Analyse de sentiment
\item Résumé automatique
\item Topic modeling\\
\end{itemize}

\noindent\textbf{Apprentissage automatique}
\begin{itemize}
\item Machine learning supervisé et non supervisé
\item Nearest neighbor
\item Régression linéaire
\item Clustering avec k-means
\item Deep learning et word2vec\\
\end{itemize}

\noindent\textbf{Aspects multilingues + privacy}
\begin{itemize}
\item Détection de la langue
\item Traduction automatique
\item Localisation
\item Respect de la vie privée
\item Anonymisation des données\\
\end{itemize}

\noindent\textbf{Évolutions et limites}
\begin{itemize}
\item Gestion de corpus hybrides
\item Veille documentaire
\item Évolution des données dans le temps
\item Impact sur les sciences humaines
\item Limites des approches quantitatives\\
\end{itemize}
