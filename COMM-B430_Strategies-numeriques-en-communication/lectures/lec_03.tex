\lecture{3}{jeudi 5 mars 2020}
\vspace{-1.2cm}

\subsection{Participative marketing}

Le magazine Time, qui élit tous les ans la personne la plus influente de l'année, a décidé d'élire le consommateur, "You", comme personne de l'année 2000. Le raisonnement derrière ce choix était d'indiquer à quel point, à l'époque, le consommateur était devenu le plus grand acteur à l'age de l'information. On dit que le consommateur est devenu un "empowered customer".\\

En effet, l'effet de masse et le bouche à oreille peuvent tout changer pour le branding d'une marque, du jour au lendemain. Mais la masse d'internautes peut être utilisée différemment et au profit de l'entreprise, en faisant du marketing participatif/collaboratif. Inclure le consommateur dans le processus de développement ou d'évolution du produit et solliciter le consommateur ou prospect pour: choisir un nouveau nom de marque, redéfinir son produit, extension de gamme, nouveau gout, participer au concept de communication, choisir ou proposer un concept, figurer dans la campagne ou le packaging, aider, stimuler les ventes,...\\

Le marketing participatif est une stratégie de marque qui a pour objectif de créer une relation avec les clients: ces derniers donnent leur avis, leurs impressions, leurs idées sur un produit ou plus généralement sur une marque. Le marketing participatif permet donc d'établir un dialogue avec les consommateurs, et par la suite de co-créer avec eux un nouveau produit, un service, un site web, une vidéo, etc. Le marketing participatif permet aux marques d'élaborer ensemble avec le consommateur un nouveau projet tout en comprenant mieux leurs besoins et leurs avis. Cela permet d'augmenter les chances de succès soit d'un lancement de produit soit d'un lancement de campagne. Cette démarche permet à l'entreprise d'engendrer du word-of-mouth et de générer du contenu sur les réseaux sociaux et d'améliorer également son référencement naturel. \\

Quels sont les avantages pour la marque?

\begin{itemize}
    \item Un consommateur qui participe est un très bon levier qui pourra influencer son entourage
    \item La marque gagne en visibilité et en notoriété grâce au bouches à oreilles. Le consommateur-participatif devient un bon ambassadeur de la marque
    \item Un autre avantage peut être la réduction des coûts puisque la participation des consommateurs est souvent apportée gratuitement ou avec des "incentives" maîtrisés
    \item Une potentielle augmentation des ventes ou tout du moins un produit qui a de grandes chances de marcher
    \item En donnant l'opportunité aux consommateurs de participer, l'entreprise peut se démarquer des autres en véhiculant des valeurs sociales, d'écoute, de dialogue et surtout de transparence.\\
\end{itemize}

Quelles sont les limites?

\begin{itemize}
    \item Risque que le message véhiculé soit déformé et que les consommateurs s'expriment négativement par rapport à la marque.
    \item Exemple: Chevrolet et une campagne sur YouTube qui proposait aux internautes de créer leur publicité pour son lancement. Certains en ont profité pour dénoncer la marque et la course au "bigger is better" et finalement la vidéo la plus vue mettait en exergue la pollution causée par les gros 4x4.
    \item Exemple: La marque Hasbro, créatrice du Monopoly, avait sollicité les internautes pour qu'ils votent pour leurs villes et ainsi remplacer les célèbres Rue de la paix et consoeurs. Un habitant de la ville de Montcuq, a eu la riche idée d'inciter les internautes à voter pour sa ville. La ville de Montcuq est arrivée première dans le classement. Néanmoins, Hasbro a décidé de ne pas suivre le vote et n'a pas repris la ville dans sa nouvelle édition prétextant que le Monopoly était un jeu familial.\\
\end{itemize}

La question éthique: Cela pose également la question du droit de la propriété intellectuelle. Les participants cèdent sans consentement leur création à l'entreprise, parfois sans contrepartie. Le créateur n'a aucun droit sur son idée, son innovation, sa production...\\

\newpage

\subsection{Exemples vus lors du cours}

\textbf{Introduction}

\begin{itemize}
    \item \textbf{MailChimp:} Cette entreprise avait réussi à acheter un segment de pub dans un des podcasts les plus populaires aux USA à l'époque, "Serial". Malheureusement, une des personnes faisant la lecture du nom de l'entreprise avait fait une erreur en le prononçant, créant une série de blagues sur internet sur ce nom. Suite à ça, MailChip a crée toute une série de campagnes différentes et improbables avec des noms marginalement similaires (MailShrimp, KaleLimp et d'autres), poussant les gens à chercher ces noms sur Google. Avec une bonne optimisation SEO, ils ont réussi à rediriger toutes les personnes cherchant ces entreprises aux concepts étranges vers un article sur leur site web. Lien Youtube: \href{https://www.youtube.com/watch?v=L4KnG4iZMnk}{Did you mean MailChimp?}
    \item \textbf{Lowe's:} Campagne de petites vidéos sur des projets DIY via Snapchat. Lien Youtube: \href{https://www.youtube.com/watch?v=09nW0CubawE}{Lowe's: In a snap}
    \item \textbf{Oreo:} Campagne pour fêter les 100 ans de la marque. 100 contenus pendant 100 jours. Lien Youtube: \href{https://www.youtube.com/watch?v=ZDSc0V3AEnk}{Oreo Daily Twist Case Study}\\
\end{itemize}


\textbf{Meaningful brands}

\begin{itemize}
    \item \textbf{Havas Budapest:} Pas un exemple de campagne, mais une vidéo d'Havas présentnat des statistiques sur les meaningful brands. Lien Youtube: \href{https://www.youtube.com/watch?v=w2m9NpQD9mE}{Meaningful Brands 2019}
    \item \textbf{Dove:} Campagne hyper connue, considérée par beaucoup comme la plus marquante au niveau des meaningful brands. Lien Youtube: \href{https://www.youtube.com/watch?v=wpM499XhMJQ}{No. 1 Dove: Campaign for Real Beauty}
    \item \textbf{Dove:} Autre campagne de Dove, se focalisant sur les préjugés que l'on a sur son propre physique versus comment les autres nous perçoivent. Lien Youtube: \href{https://www.youtube.com/watch?v=litXW91UauE}{Dove Real Beauty Sketches | You're more beautiful than you think (6mins)}
    \item \textbf{Always:} Campagne sur les stéréotypes de genre. Lien Youtube: \href{https://www.youtube.com/watch?v=XjJQBjWYDTs}{Always \#LikeAGirl}
    \item \textbf{REI:} Fermer les magasins de vêtements outdoor pendant le Black Friday pour inciter les gens à passer une journée dehors au lieu de faire du shopping de masse inutile. Lien Youtube: \href{https://www.youtube.com/watch?v=lMsxrJeJ8lU}{REI - \#OptOutside Case Study}
    \item \textbf{Volvo:} Publicité incitant les gens à non seulement utiliser des voitures électriques, mais à utiliser de l'énergie verte pour les recharger.\\
\end{itemize}


\textbf{Coach Attitude}

\begin{itemize}
    \item \textbf{NIKE:} Lien Youtube: \href{https://www.youtube.com/watch?v=WYP9AGtLvRg}{Nike: Find Your Greatness}
    \item \textbf{Nivea:} Pub papier avec un bracelet détachable qu'il est possible de donner à son enfant pour ne pas le perdre dans une plage bondée. Lien Youtube: \href{https://www.youtube.com/watch?v=nZ532wkhHYs}{"The Protection Ad" by Nivea}
    \item \textbf{Hellmann's:} Si de la mayonnaise de la marque est achetée, proposer des recettes sur base des autres ingrédients dans le caddy et fournir cette recette sur le ticket de caisse. \href{https://www.youtube.com/watch?v=h3aCVrcnFOQ}{HELLMANN'S RECIPE RECEIPT}\\
\end{itemize}

\textbf{Participative marketing}

\begin{itemize}
    \item \textbf{KBC:} Via un site web, la population en flandres pouvait indiquer quel type de marché n'était pas assez exploité dans leur ville, aidant les entrepreneurs à trouver des endroits dans lesquels s'implanter.
    \item \textbf{Danette:} "Ta tête sur Danette"
    \item \textbf{Starbucks:} "What's your starbucks idea?"
    \item \textbf{Vedett:} Fameuse campagne pour avoir ses photos sur l'étiquette de la bouteille.
    \item \textbf{Evian:} Campagne "Evian Babies". 120 millions d'utilisateurs ont uploadé une de leurs photos pour avoir un montage de sa tête sur les fameux bébés Evian.
    \item \textbf{Milka:} Campagne "The last square". Envoyer son dernier carré de chocolat à quelqu'un.
    \item \textbf{Classic21:} Campagne "The Rock Generation". Déguiser les parents et enfants en icônes de la musique rock.
    \item \textbf{McDonals Canada:} Demander aux gens dans la rue s'ils se posent des questions sur la qualité de la viande et des ingrédients, puis créer des capsules vidéos répondant à ces questions de façon transparente et honnête.
    \item \textbf{Oasis:} Publicité pour un nouvel album en contactant des artistes de rue, leur donnant des morceaux en exclusivité, puis créant une carte pour que les fans puissent entendre une version alternative de l'album avant sa sortie.
    \item \textbf{Lays:} Organisation chaque année d'un concours mondial ou régional pour le chois d'un nouveau goût de chips.
    \item \textbf{Coca-Cola:} La fameuse campagne avec les noms sur les cannettes.
    \item \textbf{PlayStation:} Créer des modèles 3D de joueurs haut-niveau d'un jeu et les introduire dedans, récompensant leur investissement dans le jeu.
    \item \textbf{Blendtec:} Campagne "Will it Blend?". Un des premiers gros buzz sur Youtube.
    \item \textbf{Apple:} Campagne "Shot on iPhone". Demander aux consommateurs d'envoyer des photos prises avec leur téléphone et les afficher en énorme format, pour vanter la qualité de la caméra sur le téléphone.\\
\end{itemize}
