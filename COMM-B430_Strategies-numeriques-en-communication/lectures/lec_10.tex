\lecture{10}{jeudi 07 mars 2020}
\vspace{-1.2cm}

\section{SEO/SEA Strategy}

\subsection{SEO}

SEO stands for "Search Engine Optimization".  (search organic). It is the process of getting traffic from the
"free," "organic," "editorial" or "natural" search results on search engines. All major search engines such as
Google, Bing and Yahoo have primary search results, where web pages and other content such as videos or
local listings are shown and ranked based on what the search engine considers most relevant to users. Payment
isn't involved, as it is with paid search ads.
The process which aims to get websites listed prominently within search engine's organic / natural search
results. Organic search results have not been paid for and are ranked by the search engine (using bots / spiders
\& algorithms) according to relevancy to the term/s entered.
Vs SEA = Search Engine Advertising (service payable).
SEO algorithm: words matter + titles matter + links matter + words used in links matter + reputation
Penguin algorithme de Google a poussé les annonceurs à optimiser les sites pour le mobile et Ipad, car Google
a changé son algorithme : les sites pas optimisés sont pas bien notés dans la recherche.
Ce qu'influence les recherches selon SEO :
- Content
- Page Titles
- Meta data, title, image \& header tags
- URL
- Google authority
- Site historical popularity/ activity
- Site age
- Quality/relevance
- Global, internal \& topical link popularity
- Site structure, architecture \& flow
- Anchor text
- Local and Google Webmaster Tools sitemap
- Content update frequency
- Text over flash and images
- Bot accessibility
Les facteurs négatives pour SEO :
- Duplicate content
- Keyword stuffing \& overuse of keywords
- Duplicate title tags \& meta data
- Slow server response time
- Link associations with Spam sites, link schemes \& link farms
- Bot blocking
- Flash-based sites*
3 facteurs clés pour réussir SEO :
- Trouver les mots clés
- User-friendly plateforme
- Backlinks = toutes les références sur votre site depuis l'extérieure

\subsection{SEA}

Google Adwords. A key part of marketing to grow your business online = buy keywords for searches to appear
your website first in the results of Google.
L'avantage : ça ouvre la possibilité de publicité pour tous les annonceurs, qu'il soit grand ou petit. 20 000 -
50 000 euros par mois pour l'annonceur moyen pour les dépenses d'AdWords, mais cela donne énormément
de traffic qualifié sur le site.
On fixe de budget maximum par clicks, permet de faire l'analyse de ces publicités pour les ajuster ensuite.

\section{UX and UI Design}

\section{The power of A/B Testing in Marketing}
