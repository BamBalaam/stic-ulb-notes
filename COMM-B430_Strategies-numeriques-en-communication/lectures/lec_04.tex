\lecture{4}{jeudi 12 mars 2020}
\vspace{-1.2cm}

\section{Gamification}

Le terme de gamification est un néologisme de langue anglaise qui désigne le fait de reprendre des mécaniques et signaux propres aux jeux, et notamment aux jeux vidéos, pour des actions ou applications qui ne sont pas des jeux. Le but de la gamification est de rendre une action plus ludique, de favoriser l'engagement de l'individu qui y participe et d'introduire éventuellement une dimension virale. Le principe de la gamification peut concerner la publicité, les applications mobiles ou l'usage d'un produit et fut très tôt utilisée dans les mécaniques promotionnelles. Dans le domaine du marketing digital le fait d'utiliser l'attribution de statuts pour encourager les participations sur un forum est un exemple classique de gamification.\\

En intégrant les attributs des jeux (points, badges, classement, défis, etc.) dans des services qui n'ont a priori rien de ludique et en mettant en place un système de récompenses, les individus cherchent à progresser en relevant des défis perçus comme ludiques ou accomplissant diverses tâches. Leur esprit de compétition ou leur altruisme est sollicité dans des situations virtuelles, parfois
liées au monde réel. Du fait de la compétition collective et de la course à la récompense, les utilisateurs sont de plus en plus motivés à utiliser ces services.\\

\section{Évolution des médias}

\boldmath
\textbf{$360^{\circ}$ VS 365}\\
\unboldmath

Une communication $360^{\circ}$ est une communication qui intègre et mobilise tous les points de contact avec le consommateur et qui va décliner une idée, un concept sur l'ensemble de ces supports, de manière cohérente. La notion de communication $360^{\circ}$ s'est développée avec le développement des supports numériques (internet, mobiles,..). Une communication $360^{\circ}$ doit avoir une cohérence sur l'ensemble des points de contacts et canaux utilisés.\\

La communication $360^{\circ}$ est nécessaire, mais plus suffisante. Il nous faut également développer une communication 365 (jours par an), c'est à dire à une mise en place de contenu qui permet de rester en contact avec le prospect, même quand la campagne de pub $360^{\circ}$ est terminée. Cette approche permet aux annonceurs de continuer à influencer le consommateur sur des supports Owned et Earned, dans une stratégie de contenu, en dehors d'achat média (campagne Paid).\\

\textbf{POE (Paid, Owned, Earned) Media}

\begin{itemize}
    \item \textbf{Paid:} désigne l'espace publicitaire acheté par la marque sur les médias digitaux ou traditionnels.
    \item \textbf{Owned:} désigne les points et supports d'exposition possédés et contrôlés par la marque. Le point central du owned media est le plus souvent le site web de marque auquel peuvent s'ajouter les comptes Facebook, Twitter, Pinterest, Instagram, Youtube,... Dans un cadre plus large, les points de vente, boutique, camionnette, la PLV, les enseignes.
    \item \textbf{Earned:} désigne le bouche à oreille et la publicité gratuite faite par les utilisateurs sur des réseaux sociaux, blogs, revues, etc.\\
\end{itemize}

Aujourd'hui, le premier choix stratégique est de bien choisir le pourcentage entre P, O et E, mais la tendance est de beaucoup plus se focaliser sur l'owned et l'earned que le paid. \\

\textbf{Exemple:} IKEA est un très bon exemple de bon équilibre POE. Le Owned est toute la communication visuelle très connue d'IKEA, toujours innovante. Le Earned est la communauté de fans autour du monde (IKEA Hackers, pages de fans sur Facebook, etc). \\

\textbf{Exemple 2:} Redbull et son investissement annuel de 30\% de son chiffre d'affaires dans le sponsoring sportif, ce qui a mené à son apogée d'Earned media: le saut de Felix Baumgartner.
