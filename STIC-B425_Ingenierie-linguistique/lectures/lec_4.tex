\lecture{4}{vendredi 21 février 2020}
\vspace{-1.2cm}

\subsection{NMT}

Dernière génération des systèmes de TA \\

\begin{itemize}
    \item 2016: GT lance le 1er système neuronal sur le web
    \item 2017: DeepL
    \item Aujourd'hui: presque tous les systèmes sur internet sont neuronaux sauf Yandex (hybride SMT/NMT) et Apertium (RBMT)
    \item 2019: Reverso (Reverso Corporate)\\
\end{itemize}

Plateforme open source: OpenNMT

Plateforme commerciale: Custom Translator (Microsoft)\\

Étude par Google en 2017 sur la qualité des systèmes NMT. Il faut garder un oeil critique sur les études de performance, surtout celles venant de l'industrie \\

\textbf{Caractéristiques}

\begin{itemize}
    \item Apprentissage profond ("deep learning")
    \item Ne reste plus au niveau des mots, mais représente le sens des mots/phrases avec des plongements (embeddings)\\
\end{itemize}

\textbf{Résultat}

\begin{itemize}
    \item Beaucoup plus de généralisations
    \item Traduit sur base du sens, plus une traduction fragmentée\\
\end{itemize}

\textbf{Étapes}

Encodage

\begin{itemize}
    \item Un réseau de neurones lit les mots un par un (les vecteurs un par un)
    \item Construit la représentation de la phrase (représentation des mots dans le contexte)\\
\end{itemize}

Décodage

\begin{itemize}
    \item Un autre réseau de neurones génére les mots cible un à un
    \item Cherche la probabilité, étant donné le plongement de la phrase en anglais, d'avoir le premier mot de la phrase en espagnol en ayant le plongement en anglais $P( X_{word} | embeddings)$\\
\end{itemize}

\textbf{Plongement lexical}

\begin{itemize}
    \item Représentation numérique distribuée
    \item Distribuée : chaque mot est situé par rapport aux autres dans un espace multidimensionnel sur base de sa distribution dans le corpus
    \item Numérique : vecteur, liste de chiffres, 1 point dans l’espace. Représentation vectorielle, un mot = un vecteur (une liste de chiffres dans l'espace)\\
\end{itemize}

\textbf{Hypothèse distributionnelle}

\begin{itemize}
    \item L. Wittgenstein (1889-1951): Le sens d’un mot vient de son usage
    \item Firth (1890-1960): "You shall know a word by the company it keeps"
    \item Harris (1958): Si deux mots apparaissent dans le même contexte, ils ont le même sens \\
\end{itemize}

\textbf{Relations sémantiques}

\begin{itemize}
    \item Plongements encodent les relation sémantiques
    \item Même distance pour les mêmes relations\\
\end{itemize}

\newpage

\textbf{Opérations sur les mots}

Semantic Arithmetics VS Composition de vecteurs \\

Semanting Arithmetics: \\

[[king]] – [[man]] + [[woman]] = [[queen]]

[[Paris]] – [[France]] + [[Allemagne]] = [[Berlin]]\\

Composition de vecteurs: \\

vecteur(king) - vecteur(man) + vecteur(woman) = vecteur(queen)\\

\textbf{SMT vS NMT: Qualité}

Traduction beaucoup plus fluide

\begin{itemize}
    \item Moins de fautes grammaticales (car il ne s'agit pas d'une traduction par segment)
    \item Plus de fautes sémantiques. Les erreurs sont sémantiquement motivées ("cappucino à la place de expresso")
    \item Plus d'omissions mais moins de non-sens
    \item A du mal avec les mots inconnus\\
\end{itemize}

\textbf{Challenges}

\begin{itemize}
    \item Phrases longues
    \item Mots rares ou inconnus
    \item Taille des données\\
\end{itemize}

\textbf{Limites informatiques}

\begin{itemize}
    \item Demande beaucoup de textes
    \item Demande des mémoires particulières (GPU, graphic processing unit)\\
\end{itemize}

\subsection{Exercices}

Détecter laquelle des traductions est issue d'un SMT et laquelle d'un NMT.\\

\textbf{Exercice 1:}\\

\begin{table}[H]
\begin{tabular}{|l|ll}
\cline{1-1}
\textbf{Source} & & \\ \hline
Strensiq (asfotase alfa) & \multicolumn{1}{l|}{Strensiq (asfotase alfa)} & \multicolumn{1}{l|}{Stensiq ( asfotase alfa)} \\ \hline
\end{tabular}
\end{table}

Réponse: La première est SMT car il a gardé le mot inconnu (Strensiq) inchangé.\\

\textbf{Exercice 2:}\\

\begin{table}[H]
\begin{tabular}{|l|ll}
\cline{1-1}
\textbf{Source} &  &  \\ \hline
\begin{tabular}[c]{@{}l@{}}An overview of Strensiq and \\ why it is authorised in the EU\end{tabular} & \multicolumn{1}{l|}{\begin{tabular}[c]{@{}l@{}}présentation de Stensiq et des raisons\\ pour lesquelles elle est agréée dans l’UE\end{tabular}} & \multicolumn{1}{l|}{\begin{tabular}[c]{@{}l@{}}Une vue d’ensemble de Strensiq et\\ pourquoi il est autorisé dans l’UE\end{tabular}} \\ \hline
\end{tabular}
\end{table}

Réponse: Reformulation de la phrase dans la première traduction (NMT). Fluide mais l'apparition du féminin ne fait pas de sens. SMT garde l'ordre des mots.\\

\newpage

\textbf{Exercice 3:}\\

\begin{table}[H]
\hspace{-2cm}
\begin{tabular}{|l|ll}
\cline{1-1}
\textbf{Source}  &  & \\ \hline
\begin{tabular}[c]{@{}l@{}}Strensiq is a medicine used long-term \\ to treat patients with hypophosphatasia \\ that started in childhood.\end{tabular} & \multicolumn{1}{l|}{\begin{tabular}[c]{@{}l@{}}Strensiq est un médicament utilisé pour le\\ traitement à long terme des patients \\ atteints de hypophosphatasia \\ que prescrit chez l’enfant.\end{tabular}} & \multicolumn{1}{l|}{\begin{tabular}[c]{@{}l@{}}Stensiq est un médicament utilisé\\ à long terme pour traiter les patients\\ atteints d’hypophosphite d’hyposie\\ qui ont commencé à l’enfance.\end{tabular}} \\ \hline
\end{tabular}
\end{table}

Réponse: La première traduction est SMT, la deuxième NMT. Encore une fois, NMT a essayé de traduire un mot inconnu (hypophosphatasia), tandis que SMT l'a laissé inchangé. SMT a également traduit de façon incorrecte la fin de la phrase.\\

\textbf{Exercice 4:}\\

\begin{table}[H]
\begin{tabular}{|l|ll}
\cline{1-1}
\textbf{Source} & & \\ \hline
How is Strensiq used? & \multicolumn{1}{l|}{Comment cette entité est-elle utilisée?} & \multicolumn{1}{l|}{Comment Strensiq est-il utilisé?} \\ \hline
\end{tabular}
\end{table}

Réponse: NMT pour la première traduction, SMT pour la deuxième. NMT ne connaissant pas le mot inconnu l'a traduit par "entité".\\

\textbf{Exercice 5:}\\

\begin{table}[H]
\hspace{-1cm}
\begin{tabular}{|l|ll}
\cline{1-1}
\textbf{Source} & & \\ \hline
\begin{tabular}[c]{@{}l@{}}For more information about using\\ Strensiq, see the package leaflet\end{tabular} & \multicolumn{1}{l|}{\begin{tabular}[c]{@{}l@{}}Pour de plus amples informations\\ sur l’utilisation de Stensiq, voir la notice\end{tabular}} & \multicolumn{1}{l|}{\begin{tabular}[c]{@{}l@{}}Pour plus d’informations sur utilisant \\ Strensiq, voir la notice\end{tabular}} \\ \hline
\end{tabular}
\end{table}

Réponse: NMT pour la première traduction, SMT pour la deuxième. SMT a fait une traduction littérale de l'expression "about using", la traduisant en "sur utilisant".\\
