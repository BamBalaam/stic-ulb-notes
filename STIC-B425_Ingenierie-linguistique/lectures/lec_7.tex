\lecture{7}{mercredi 18 mars 2020}
\vspace{-1.2cm}

\section{Aspects syntaxiques}

Skip to 15:25 for interesting part

Syntaxe

    Du grec ancien %σύνταξις (súntaxis) qui signifie « arrangement » ou « mise en ordre »

    Façon dont les mots se combinent pour former des phrases structurées

    Unité syntaxique = syntagme (groupe de mots)

Grammaire context-free (CFG)

    Simplification du langage naturel mais plus flexible qu'une grammaire régulière

    Permet de modéliser les relations entre constituants

    Ensemble de règles pour organiser symboles grammaticaux et unités lexicales

    Génération et analyse (parsing)

Grammaire générative

    Un sous-langage est défini par l'ensemble des phrases générées par la grammaire

    Puissance générative : une infinité de phrases peut être générée par un ensemble fini de règles

    Équilibre à trouver entre sous-génération (silence) et sur-génération (bruit)

Treebanks

    Corpus de textes annotés avec leurs structures syntaxiques correspondantes

    Révolution pour le TAL !

    Penn Treebank (1992)
    French Treebank (1997)

Parsing (analyse grammaticale)
    Tâche qui consiste à reconnaître une phrase et à y assigner une structure syntaxique

    Applications diverses
        correction grammaticale
        traduction automatique
        question answering

    Recherche de l'arbre correct parmi l'ensemble des arbres possibles (cf. treebanks)

Stratégies de recherche
    « Marie voit un chat. »

    Deux contraintes
        grammaticale : former une phrase (symbole P)lexicale : utiliser les quatre mots dans l'ordre
    Deux approches
        top-down : en partant de l'objectif à atteindre
        bottom-up : en partant des données à utiliser


(.....)

Exercice 4

